\documentclass[11pt]{article}

\usepackage{hyperref}
\setlength{\oddsidemargin}{0pt}
\setlength{\topmargin}{0pt}
\setlength{\textheight}{8.5in}
\setlength{\textwidth}{6.5in}
\usepackage{graphics}
\usepackage{graphicx}
\usepackage{color}
\usepackage{framed}

\begin{document}
\title{Seven-Letter Word\\Group 1 Final Report}

\author{
	Nipun Arora (nipun@cs.columbia.edu)
 \and Manuel Entrena Casas (mae2135@columbia.edu)
 \and Ben Warfield (bbw2108@columbia.edu)}

\date{December 14, 2009}
\maketitle

\newpage
\setcounter{tocdepth}{2}
\tableofcontents
\newpage

\section{ Introduction }
In which we say what everybody already knows.  Paragraph text separated by double newlines.

\subsection{Word List}
\subsection{Maximizing Seven-Letter Words}

\subsection{Minimizing Costs}

We can list things in a list thus:
\begin{enumerate}
\item Firstly
\item Secondly
\item Finally
\end{enumerate}

\section{Framework}

\subsection{Player Architecture}

\subsection{A Priori Algorithm} % BEN
\subsubsection{As originally designed}

Cite Agrawal '94.  Note originally implemented for COMS E6111.

\subsubsection{Modifications for Word Context}

Duplicate letters.  No, I'm not going to explain how I initially screwed it up, just how I eventually fixed it.

\subsection{Probability and Reachability Calculations} %BEN 

Some words have probability 0--we can generally predict that.


Yay, \LaTeX!
% good thing we're not using Google, after all...
$$\prod_{i=1}^{26}{n_{i}\choose r_{i}}*\left( \sum_{i}r_{i}\right)!$$

Note that the factorial at the end for permutation is useless, since we're not actually doing the rest of the math, but it's useless at a constant rate for all the words that matter (seven-letter words will all have draw sequences of the same length, just differing likelihoods).

\section{Bidding Strategies}

\subsection{When to Bid}

Our high-level bidding strategy is simple: we query the {\it a priori}-generated data mine using the set of letters we have in our rack at the moment, and determine what (if any) seven-letter words include them.  If any of these seven-letter words remains reachable (no matter how unlikely the possibility), we will bid only on letters that help us reach one of those words.
If no such word exists, we simply bid incrementally: we find the best-scoring word we can make with our existing rack and the best-scoring word we could make if we acquired the letter currently up for bid, then bid the difference (if positive) between the two scores.  This strategy, much discussed in class, has a worst-case behavior of zero gain, and (in contrast to any strategy that takes into account what other non-seven-letter words we might hypothetically form with other letters) is extremely simple to implement.

\subsection{Gain-based strategies}

Our initial approach to bidding (once we gave up on the ``bid 0 and make the best of it'' strategy) attempted to bid based on how much closer acquiring a particular letter would bring us to having a seven-letter word.  

This strategy is intuitively reasonable: if we succeed in every bid, we will have paid a cost no higher than the benefit (50 points) that we expect from forming a seven-letter word.

\subsubsection{Word Count}

Which we didn't actually implement, but is obvious to talk about.

\subsubsection{Summed Word Probabilities}

And ranked.

With replacement.

\subsubsection{Word Draw Possibilities}

The immediate problem here is that the possibilities count will virtually always go down when you acquire a new letter.  The best letters will reduce the count by only a small amount; the worst may reduce it to zero.

Cutoff-based bidding, still based on the tile value.

Cutoff-based bidding, based on Manuel's  % kind of optimistic ;-)
estimate of how much things were worth. 

\subsubsection{Summary}

These strategies successfully rescued our player from stinginess, allowing us to form seven-letter words with increasing frequency.  The cost they imposed, however, was disproportionate to the benefit: our player generally averaged significantly positive scores, but would fall far behind those of other groups.

\subsection{Loss-based strategies}

The obvious alternative to gain-based bidding is loss-based bidding: each time a letter is placed up for bid, we estimate how badly our chances of reaching a 7-letter word would be hurt by not acquiring it, then multiply that penalty by the benefit we would forgo (again, 50 points).  

\subsubsection{Basic Calculation}

Draw possibilities now - draw possibilities if this letter is removed from the bag.

\subsubsection{Opportunity Cost}

This strategy brings us back face to face with the problem we declined to solve earlier: how does the probability of a given sequence of letters being reached change with the number of remaining auctions?  

Our solution was sort of unsatisfactory, but our efforts to improve on it largely failed.

\subsubsection{Market Rates}

In principle, if we are able to form a perfect estimate of the opportunity cost of passing on one auction, as well as the penalty for failing to acquire each letter, we should be able to perfectly calculate the value (in expectation) of each letter, and our player should perform well, at least on average.  However, since our estimates are imperfect, and we may be in competition with other players, we also attempt to keep track of how our bids compare to the ``market rate'' for each letter: if we are losing out on our bids consistently, we may be systematically underestimating the value of letters.

We considered a number of tactics for adjusting our bidding to avoid falling behind on letter acquisition:

\begin{enumerate}
\item{} periodically boosting our bid by some fixed amount or factor, to avoid consistently being outbid (and confound any opposing-team predictors)
\item{} keeping track of the number of letters acquired and the number of bidding rounds, and boosting our bid if we are behind the ``average'' rate of letter acquisition (that is, if more than $1/7$ of letters have been auctioned off, but we have only 1 or 0 letters on our rack).
\item monitoring our success rate on recent bids, and boosting the bid by a fixed factor if we have been losing bidders some number of times in a row
\item extending the previous strategy by maintaining (or continuing to escalate) the boost factor until we succeed on a bid.
\end{enumerate}

We evaluated all of these approaches in mini-tournaments, against each other and the most recent version of players from the other groups: our final conclusion was that the penalty for underbidding was not crippling us in any case, so a relatively conservative modification to our bidding would suffice.  Our final player tracks the success or failure of each non-zero bid, and boosts its bids by a factor of 2 if the last three or more non-zero bids have been unsuccessful.


\section{Tournament Results} % NIPUN

\section{Issues Encountered/Observations/Lessons Learned}

Draw possibilities code had a bug early on which may have compromised some comparisons we made, but we are confident we ended up with a good solution.

\section{Possible Extensions}

Well, it might be interesting to have more explicit probability calculations for the late-game situation where you're trying to decide between a profitable increment and an unlikely seven-letter word.

You can't get an incredibly well-defined probability of getting what you need, but you can set an upper bound on it, which is all you need to see if the increment is worth more in expectation than trying for the 50-point bonus.  Very small enhancement, though.



\section{Acknowledgements}

Thanks for the acronym-eliminated word list, Jon.  Also the MySQL database for analysis purposes.

Also thanks to Professor Gravano for assigning the Agrawal paper and implementation for 6111.

And thanks to Dr. Virginia Warfield, Ph.D., for getting Ben interested in probability about 20 years ago.

\section{Contributions}



\section{Conclusion}
\end{document}
